
\documentclass{article}
\usepackage{currvita}
\usepackage[english]{babel}
\usepackage[utf8]{inputenc}
\usepackage[dvips]{epsfig}
\usepackage{charter}
\usepackage{hyperref}

\oddsidemargin  0.5cm  % Ancho Letter 21,59cm
\evensidemargin 0.5cm  % Alto  Letter 27,81cm
\textwidth      15.5cm
\topmargin      0cm
\textheight     20cm
\parindent      2cm
\parskip        2ex

\begin{document}

\setlength{\cvlabelwidth}{45mm}

\begin{cv}{David García Garzón. Curriculum Vitae}

%\includegraphics[scale=.23]{http://canvoki.net/bichos-low.png}
\begin{cvlist}{Personal Data}
\item[Full name:] David García Garzón
\item[Born] 10th May 1975, Barcelona, Spain
\item[Nationality] Spain
\item[E-mail] david.garcia at upf.edu
\end{cvlist}

\begin{cvlist}{Summary}
\item[] I've been for a decade a profesor and a researcher on Audio and Music Software Engineering
and an specialist in Agile and Free/Libre Open Source Software Development methods.
I have been teaching mostly at Universitat Pompeu Fabra, but also other universities, an activity that I enjoy a lot.
Currently I am involved in several cooperatives of the Social Economy,
currently working at Som Energia and GuifiBaix SCCL,
a local Guifi.net deployer.
I also have many contacts and bounds to many other entities such as
Som Connexio, Mes Opcions, Coop57, Pam a Pam...

Although my formal education is mostly technical, my interests go on a wide range of topics
from Biology and Physics to Politics, History, Economy and Philosophy.
Indeed I have been always interested the social part of Computer Science,
Free Software communities, its internal dynamics, and
their struggle to reach the masses in rivalry with powerfull monopolies.
After the 15M upraising in Barcelona, I decided to focus my efforts
on constructing, supporting and promoting social economy iniciatives
applying lessons learnt in the software community and, indeed learning a lot.

\end{cvlist}

\begin{cvlist}{Interests}
\item[] Cooperativism and Social Economy, Music software engineering, Music information retrieval, 3D acoustics simulation, 3D audio production and playback, Emergent evolutionary systems, Free and Open Source Software, Agile Software Project Management, Teaching
\end{cvlist}



\begin{cvlist}{Language skills}
\item[Spanish] Native speaker
\item[Catalan] Native speaker
\item[English] Fluent (B2)

\end{cvlist}


\begin{cvlist}{Education}

\item[2007 - Unfinished]
{\bf PhD} in {\bf Information, Communication and Audiovisual Media Technologies}
coursed at {\bf Universitat Pompeu Fabra}.
{\em Proposal: Relating audio and 3D scenarios in audiovisual productions.}
\item[2006 - 2007]
{\bf Master} in {\bf Information, Communication and Audiovisual Media Technologies}
coursed at {\bf Universitat Pompeu Fabra}.
{\em Thesis: Visual Prototyping for Audio Applications.}
Topics: Audio Software Engineering, Frameworks, Domain Specific Languages, Prototyping tools
\item[1997 - 2002]
{\bf Engineering} in {\bf Computer Science}
coursed at {\bf Enginyeria la Salle, Universitat Ramon Llull}.
{\em Final Career Project: XML support for an audio framework.}
Topics: XML, C++ template metaprogramming, Audio processing, Frameworks design
\item[1993 - 1997]
{\bf Technical Engineering} in {\bf Computer Science}
coursed at {\bf Enginyeria la Salle, Universitat Ramon Llull}.
{\em Final Career Work: Bioscena, an evolutionary environment with interaction among individuals.}
Topics: Artificial Life, Genetic algorithms, Advanced OO Design
\item[1988 - 1993]
{\bf BUP/COU (Bachelor)} in {\bf Scientific/Sanitary}
coursed at {\bf Col·legi Llor (Sant Boi de Llobregat)}.
{\em }
\end{cvlist}

\begin{cvlist}{Awards and Honors}

\item[2014]
{\bf Special Mention}
at {\bf Best idea or project 2014, Ajuntament del Prat de Llobregat}
with the work {\em GuifiBaix business plan}.
\item[2009]
{\bf Outstanding results}
at {\bf Pilot Bologna Plan Course Migration at UPF}
with the work {\em Software Engineering I}.
\item[2006]
{\bf Winner}
at {\bf ACM Multimedia 2006 Open Source Software Competition}
with the work {\em CLAM framework}.
\end{cvlist}

\begin{cvlist}{Professional Experience}

\item[Jan 2015-Now]
{\bf Developer} at
{\bf Som Energia SCCL.}\\
Creating and organizing an agile IT team nearly from scratch
dealing and coordinating multiple external partners
managing multiple sources of priorities in an horizontal organization
working with a plethora of Python and Javascript technologies:
OpenERP, Flask, Django, Angular, Mithril, Bootstrap, MDW, Asterisk...


\item[Jun 2013-Now]
{\bf Founder and partner} at
{\bf GuifiBaix SCCL.}\\
Inception and execution of a business model for the cooperative
Wired and wireless network deployments, and
Web application development (P2P, Cloud services, streaming...).
Programming bussiness logic.


\item[Jan 2000-Now]
{\bf Core developer} at
{\bf CLAM (C++ Library for Audio and Music).}\\
I've take part of the team of core developers of the CLAM framework.
CLAM has become a framework of reference in multimedia and audio.
It won the 2006 ACM Award to the Best Multimedia Open Source Software
and was a featured project for 2007 and 2008 editions of the Google Summer of Code.


\item[Jan 2013-Jun 2013]
{\bf Volunteer teacher} at
{\bf Associació Impulsant.}\\
Empowering digitally excluded women by teaching them on how to use social networks and open source technologies.


\item[Sep 2002-Sep 2012]
{\bf Professor} at
{\bf Universitat Pompeu Fabra.}\\
Teaching Software Engineering (Requirements gathering, UML
Design Patterns, Test Driven Development, Refactoring
Agile Methodologies, C++) and
Programming III (Object oriented programming, Java).


\item[Jun 2007-Nov 2012]
{\bf Researcher} at
{\bf Barcelona Mèdia.}\\
Design and develop real-time systems for 3D audio.
The system became the basis for a commercial start-up (ImmSound).
It was deployed in cinemas world wide, until Dolby decided to buy the technology.
My contributions include defining the processing architecture as well as the application
algorithm transcription and optimization, and junior training.


\item[Dec 2005-Dec 2009]
{\bf Co-founder and partner} at
{\bf BMAT, Barcelona Music and Audio Technologies.}\\


\item[Sep 2002-Jun 2004]
{\bf Professor} at
{\bf Fundació UPC.}\\
Course 'Software Engineering Tools and Methodologies on
Free Software Platforms' within the Free Software Master.


\item[Sep 2000-Dec 2006]
{\bf Research Assistant} at
{\bf Music Technology Group of the Universitat Pompeu Fabra.}\\
Providing group wide support for audio software engineering
dealing with technology advising, software quality assurance
software integration, software architecture, graphic interfaces and packaging.
Involved in projects such as CLAM, Agnula, Cuidado, and Simac.
Specialized in Free Software and Music Information Retrieval.


\item[Feb 2005-Apr 2005]
{\bf Visitor Researcher} at
{\bf Electronic Engineering Department at Queen Mary University of London.}\\


\item[Aug 2000-Dec 2003]
{\bf Senior Programmer and Analyst} at
{\bf Cards Engineering Spain, S.L..}\\
Developing intranet applications for industrial engineering environments.
Deploying the development environment (CVS, Mantis, Automated tests...).
Involved PHP, unix shell scripting and dealing multiple UNIX flavors.


\item[Sep 1998-Jun 1999]
{\bf Teacher} at
{\bf FUNITEC.}\\
Organizing and teaching an Ongoing Education Course on Computer Music at Enginyeria la Salle. (42 teaching hours)

\item[Jul 1997-Dec 1997]
{\bf Junior Programmer} at
{\bf NexTReT.}\\
Working at client site, EDS Iberia, on the ticketing section:
Legacy code maintenance involving AIX, C, DB’s and
ticket printer low level programming.
Quality control and assurance on a developing system involving Oracle technologies.


\item[Sep 1994-Jun 1997]
{\bf Collaborator and teacher assistant} at
{\bf Software Technology section of the Departament d'Informàtica of Enginyeria La Salle.}\\
Preparing written materials (a Smalltalk-80 manual).
Teaching courses on Advanced C.
Teaching laboratory sessions of Programming I course.
Educational intensification on Programming II.
Adapting a Genetic Algorithms tool to solve permutation problems (TSP).

	
\end{cvlist}

\begin{cvlist}{Publications}

\item[] {\sc Tim Schmele, David García-Garzón, Umut Sayin, Davide Scaini and Daniel Arteaga} 2013.
'{\em Layout Remapping Tool for Multichannel Audio Productions}'
134th Audio Engineering Society Convention; May 2013; Parma, Italy

\item[] {\sc David Garcia Garzón and Xavier Serra Roman} 2013.
'{\em IPyCLAM Enpowering CLAM with Python}'
11th Linux Audio Conference 2013; University of Music and Performing Arts; Graz, Austria; May 2013

\item[] {\sc D. Garcia, D. Arteaga,  J.  Usher, T. Mateos} 2010.
'{\em Determining a room geometry from its impulse response}'
Presented at Internoise10, Lisbon, June 2010.

\item[] {\sc Bailer, W. Arumi, P. Mateos, T. Garriga, A. Durany, J. and Garcia, D.} 2009.
'{\em Estimating 3D Camera Motion for Rendering Audio in Virtual Scenes}'
5th European Conference on Visual Media Production, 2008.

\item[] {\sc Jun Wang, Xavier Amatriain, David Garcia Garzón, Jinlin Wang} 2009.
'{\em Combining multi-level audio descriptors via web identification and aggregation}'
Presented at World Wide Web Conference'09, Developers Track, Madrid

\item[] {\sc P. Arumi, D. Garcia, T. Mateos, A. Garriga and J. Durany.} 2008.
'{\em Real-time 3D audio for digital cinema}'
ASA Conference ACOUSTICS'08 Paris.

\item[] {\sc Garcia, D.} 2007.
'{\em Visual Prototyping of Audio Applications}'
Master Thesis, Master Program in Information, Communication and Audiovisual Media. Advisors: Xavier Serra and Xavier Amatriain. Department of Information and Communication Technologies, Universitat Pompeu Fabra. Barcelona, September 2007.

\item[] {\sc Olaiz, N. Arumí, P. Mateos, T. García, D.} 2009.
'{\em 3D-Audio with CLAM and Blender's Game Engine}'
Proceedings of the 7th International Linux Audio Conference (LAC09); April 2009; Parma, Italy.

\item[] {\sc Arumí, P. Amatriain, X. García, D.} 2008.
'{\em A Framework for Efficient and Rapid Development of Cross-platform Audio Applications}'
ACM Multimedia Systems Journal; 14(1) June 2008

\item[] {\sc García, D. Arumí, P. Amatriain, X.} 2007.
'{\em Visual prototyping of audio applications}'
Proceedings of Linux Audio Conference 2007; Berlin; Germany

\item[] {\sc Arumí, P. García, D. Amatriain, X.} 2006.
'{\em A Data Flow Pattern Language for Audio and Music Computing}'
Proceedings of Pattern Languages of Programs 2006; Portland, Oregon

\item[] {\sc Amatriain, X. Arumí, P. García, D.} 2006.
'{\em CLAM: A Framework for Efficient and Rapid Development of Cross-platform Audio Applications}'
Proceedings of ACM Multimedia 2006; Santa Barbara, CA

\item[] {\sc Arumí, P. Sordo, M. García, D. Amatriain, X.} 2006.
'{\em Testfarm, una eina per millorar el desenvolupament del programari lliure}'
Proceedings of V Jornades de Programari Lliure; Barcelona

\item[] {\sc García, D. Arumí, P. Amatriain, X.} 2006.
'{\em Extracció d'acords amb l'Anotador de Música de CLAM}'
Proceedings of V Jornades de Programari Lliure; Barcelona

\item[] {\sc Amatriain, X. Massaguer, J. García, D. Mosquera, I.} 2005.
'{\em The CLAM Annotator: A Cross-platform Audio Descriptors Editing Tool}'
Poster presented at 6th International Conference on Music Information Retrieval; London, UK

\item[] {\sc Cano, P. Koppenberger, M. Wack, N. G. Mahedero, J. Masip, J. Celma, O. García, D. Gómez, E. Gouyon, F. Guaus, E. Herrera, P. Massaguer, J. Ong, B. Ramírez, M. Streich, S. Serra, X.} 2005.
'{\em An Industrial-Strength Content-based Music Recommendation System}'
Proceedings of 28th Annual International ACM SIGIR Conference; Salvador, Brazil

\item[] {\sc Cano, P. Koppenberger, M. Wack, N. G. Mahedero, J. Aussenac, T. Marxer, R. Masip, J. Celma, O. García, D. Gómez, E. Gouyon, F. Guaus, E. Herrera, P. Massaguer, J. Ong, B. Ramírez, M. Streich, S. Serra, X.} 2005.
'{\em Content-based Music Audio Recommendation}'
Proceedings of ACM Multimedia 2005; Singapore, Singapore

\item[] {\sc Herrera, P. Celma, O. Massaguer, J. Cano, P. Gómez, E. Gouyon, F. Koppenberger, M. García, D. G. Mahedero, J. Wack, N.} 2005.
'{\em Mucosa: a music content semantic annotator}'
Proceedings of 6th International Conference on Music Information Retrieval; London, UK

\item[] {\sc Arumí, P. García, D. Amatriain, X.} 2003.
'{\em CLAM, Una llibreria lliure per Audio i Música}'
Proceedings of II Jornades de Software Lliure; Barcelona

\item[] {\sc Celma, O. Gómez, E. Janer, J. Gouyon, F. Herrera, P. García, D.} 2004.
'{\em Tools for Content-Based Retrieval and Transformation of Audio Using MPEG-7: The SPOffline and the MDTools}'
Proceedings of 25th International AES Conference; London, UK

\item[] {\sc García, D.} 2002.
'{\em Suport de XML/MPEG-7 per una llibreria de processat d'àudio i música.}'
Enginyeria La Salle. Barcelona

\item[] {\sc Amatriain, X. de Boer, M. Robledo, E. García, D.} 2002.
'{\em CLAM: An OO Framework for Developing Audio and Music Applications}'
Proceedings of 17th Annual ACM Conference on Object-Oriented Programming, Systems, Languages and Applications. Seattle, WA, USA

\item[] {\sc García, D. Amatriain, X.} 2001.
'{\em XML as a means of control for audio processing, synthesis and analysis}'
Proceedings of MOSART Workshop on Current Research Directions in Computer Music. Barcelona
	
\end{cvlist}


\begin{cvlist}{Technical skills}

\item[Programming Languages]
	I have a deep knowledge on programming and scripting languages
such  as C++, Python, PHP, Smalltallk, Java and Bash.


\item[Desktop/Mobile Programming]
	I am quite a fan of Qt technologies for building user interfaces.
I also have used the lesser Tcl/Tk, Gtk and Fltk.


\item[Web Programming]
	I am experienced with HTML5/CSS3, XML, XPath, XSD, JSON, YAML... standards.
I worked with different Javascript frontend frameworks such as JQuery, Mithril, Angular, React...
I am experienced with backend programming, both in PHP and Python using frameworks such as Flask and Django.
I managed basicly and deployed apps on Apache, Nginx and uWSGI.


\item[Operating systems]
	I've been working with several Linux flavors, mostly with Debian, SuSe, Fedora and Ubuntu.
I tend to forget anything I learnt about Windows, but I could manage to make it work when needed.
I just have very slight knowledge of MacOsX, just what I needed to barely have my Qt/POSIX apps working on it.


\item[Databases]
	I am an experienced SQL programmer.
I have used ORM's such as ADOdb, DAO, ODBC, PonyORM, SQLAlchemy.
I have experience administrating MySql and Postgresql databases.
I have theoretical knowledge on the internal implementation of distributed database managers.


\item[Networks]
	I have been messing with the Guifi.net project dealing with routers, wireless antennas, IP6 and qMp (mesh configurations).


\item[Sound]
	I've been involved on the development of an audio and music framework
involving advanced real-time signal processing and
multiplatform audio hardware programming.
I've got a lot of experience analyzing existing audio software to port, to enhance or to optimize it.
I have experience setting up MIDI networks.
I have a deep knowledge of the Linux audio infrastructure (ALSA, Jack...)
I have have experience programing plugins and hosts for different plugin platforms: Ladspa, Lv2, Vst, and AudioUnits.


\item[3D Programming]
	I have experience programing the OpenGL API.
I experience on programming raster, shading and texturization routines.
I have some notions on modeling and animating scenes.
I have experience on modeling languages such as PovRay.


\item[2D Graphics]
	I have experience using Gimp (PaintShop equivalent) and Inkscape (Freehand/Illustrator equivalent).
I used them for both artistic and professional purposes.


\end{cvlist}

\begin{cvlist}{Courses}

\item[Jun 2014]
	{\bf Project development for Co-operatives } provided by {\bf Federació de Cooperatives de Treball de Catalunya} (12 hours)

\item[Jun 2014]
	{\bf Treasury Management for Co-operatives } provided by {\bf Federació de Cooperatives de Treball de Catalunya} (12 hours)

\item[May 2014]
	{\bf Creativity and innovation in social economy enterprises } provided by {\bf Federació de Cooperatives de Treball de Catalunya} (12 hours)

\item[May 2014]
	{\bf Being a Co-operative Member } provided by {\bf Federació de Cooperatives de Treball de Catalunya} (12 hours)

\item[Oct 2013]
	{\bf Theatre Techniques for the Classroom } provided by {\bf CQUID Universitat Pompeu Fabra} (12 hours)

\item[Jun 2013]
	{\bf Collaborative Moore Method for Math teaching } provided by {\bf CQUID Universitat Pompeu Fabra} (2 hours)

\item[Feb 2013]
	{\bf Educational methodologies: teaching large groups } provided by {\bf CQUID Universitat Pompeu Fabra} (6 hours)

\item[Feb 2013]
	{\bf Educational methodologies: seminars, tutorship and workshops } provided by {\bf CQUID Universitat Pompeu Fabra} (6 hours)

\item[Dec 2012]
	{\bf Reflexive learning as methodology to promote student motivation and creativity } provided by {\bf CQUID Universitat Pompeu Fabra} (3 hours)

\item[Nov 2012]
	{\bf Educational and scientific dissemination, copyright, Creative Common licenses and repositories } provided by {\bf CQUID Universitat Pompeu Fabra} (3 hours)

\item[May 2007]
	{\bf Pedagogical Support Sessions for the Pilot Course Migration to the Bologne Plan } provided by {\bf CQUID Universitat Pompeu Fabra} (80 hours, aprox)

\end{cvlist}

\begin{cvlist}{Portfolio}
\item[CLAM (C++ Library of Audio and Music)]
Being one of the core developers I have been involved in most of the aspects of that awarded free software project. XML support, GUI, audio back-end's, architecture, reflection, processing algorithms, quality assurance...
\footnote{\href{http://clam-project.org}{http://clam-project.org}}
\item[My technical blog and web page]
Daily discussion on technical stuff I work on My technical web page contains some tutorials and several minor projects not listed here including an SConstruct tool to support Qt4, an UML Use Case diagram generator, a continuous integration efficiency tracker...
\footnote{\href{http://canvoki.net/coder}{http://canvoki.net/coder}}
\item[Personal GitHub page]
Several projects I maintain in GitHub and public contributions to other projects hosted there.
\footnote{\href{http://github.com/vokimon}{http://github.com/vokimon}}
\item[WiKo (The wiki compiler)]
It is a Python script to generate webs, scientific articles and blogs from files in wiki format.  I developed this tool to ease the maintaining of my own web sites. Since them it has become a community project at sourceforge and we have successfully used it in several projects I was involved in. Some nice features are: produces multiple output (pdf/html) from a single input (wiki), supports BiBTeX and LaTeX formulae, imports blogs from blogger, generates image galleries...
\footnote{\href{http://wiko.sourceforge.net}{http://wiko.sourceforge.net}}
\item[TestFarm]
TestFarm is a client server system for continous integration. It can be thought as a BuilBot alternative, more focused on voluteers based projects with no 24/7 compliation farms available. After co-mentoring with Pau Arumí the student that started it, Mohamed Sordo, I have been mantaining it and in 2012 I did a deep rewrite for 2.0 version.
\footnote{\href{https://github.com/clam-project/testfarm}{https://github.com/clam-project/testfarm}}
\item[PyVitae]
PyVitae is a python script to define curriculum vitae using a simple  text based syntax and generating several outputs including html and pdf. Like WiKo, it follows a write-once-generate-many principle. This curriculum has been generated using it.
\footnote{\href{https://github.com/vokimon/vitae}{https://github.com/vokimon/vitae}}
\item[Bioscena]
A computer simulation of evolutive biological systems developed as my Technical Engineering Project. A multiplatform implementation in C++, highly configurable to simulate different scenarios, varying physical and biological rules. You can play with genic expresion mechanisms and phenotipical interactions including codon decoding, intron and exon zones, promoter zones, multiple chromosomes several kinds of mutations... Simulations provide insight of emergent life features such as self regulation of mortality and aging, family bounds, depredation, symbiosy, and adaptation to periodical environment changes (day/night, seasons, longer cycles).
\footnote{\href{https://github.com/vokimon/bioscena}{https://github.com/vokimon/bioscena}}

\end{cvlist}

\cvplace{Barcelona}

\vspace{2cm}

\end{cv}
\end{document}
